
% /©
%
% Copyright 2006-2017   Will Robertson <will.robertson@latex-project.org>
% Copyright 2010-2013 Philipp Stephani <st_philipp@yahoo.de>
% Copyright 2012-2015     Khaled Hosny <khaledhosny@eglug.org>
%
% This package is free software and may be redistributed and/or modified under
% the conditions of the LaTeX Project Public License, version 1.3c or higher
% (your choice): <http://www.latex-project.org/lppl/>.
%
% This work is "maintained" by Will Robertson.
%
% ©/

\section{Documenting maths support in the NFSS}

In the following, \meta{NFSS decl.} stands for something like |{T1}{lmr}{m}{n}|.

\begin{description}
\item[Maths symbol fonts] Fonts for symbols: $\propto$, $\leq$, $\rightarrow$

\cmd\DeclareSymbolFont\marg{name}\meta{NFSS decl.}\\
Declares a named maths font such as |operators| from which symbols are defined with \cmd\DeclareMathSymbol.

\item[Maths alphabet fonts] Fonts for {\font\1=cmmi10 at 10pt\1 ABC}\,–\,{\font\1=cmmi10 at 10pt\1 xyz}, {\font\1=eufm10 at 10pt\1 ABC}\,–\,{\font\1=cmsy10 at 10pt\1 XYZ}, etc.

\cmd\DeclareMathAlphabet\marg{cmd}\meta{NFSS decl.}

For commands such as \cmd\mathbf, accessed
through maths mode that are unaffected by the current text font, and which are used for
alphabetic symbols in the \ascii\ range.

\cmd\DeclareSymbolFontAlphabet\marg{cmd}\marg{name}

Alternative (and optimisation) for \cmd\DeclareMathAlphabet\ if a single font is being used
for both alphabetic characters (as above) and symbols.

\item[Maths `versions'] Different maths weights can be defined with the following, switched
in text with the \cmd\mathversion\marg{maths version} command.

\cmd\SetSymbolFont\marg{name}\marg{maths version}\meta{NFSS decl.}\\
\cmd\SetMathAlphabet\marg{cmd}\marg{maths version}\meta{NFSS decl.}

\item[Maths symbols] Symbol definitions in maths for both characters (=) and macros (\cmd\eqdef):
\cmd\DeclareMathSymbol\marg{symbol}\marg{type}\marg{named font}\marg{slot}
This is the macro that actually defines which font each symbol comes from and how they behave.
\end{description}
Delimiters and radicals use wrappers around \TeX's \cmd\delimiter/\cmd\radical\ primitives,
which are re-designed in \XeTeX. The syntax used in \LaTeX's NFSS is therefore not so relevant here.
\begin{description}
\item[Delimiters] A special class of maths symbol which enlarge themselves in certain contexts.

\cmd\DeclareMathDelimiter\marg{symbol}\marg{type}\marg{sym.\ font}\marg{slot}\marg{sym.\ font}\marg{slot}

\item[Radicals] Similar to delimiters (\cmd\DeclareMathRadical\ takes the same syntax) but
behave `weirdly'.
\end{description}
In those cases, glyph slots in \emph{two} symbol fonts are required; one for the small (`regular') case,
the other for situations when the glyph is larger. This is not the case in \XeTeX.

Accents are not included yet.

\paragraph{Summary}

For symbols, something like:
\begin{Verbatim}
\def\DeclareMathSymbol#1#2#3#4{
  \global\mathchardef#1"\mathchar@type#2
    \expandafter\hexnumber@\csname sym#2\endcsname
    {\hexnumber@{\count\z@}\hexnumber@{\count\tw@}}}
\end{Verbatim}
For characters, something like:
\begin{Verbatim}
\def\DeclareMathSymbol#1#2#3#4{
  \global\mathcode`#1"\mathchar@type#2
    \expandafter\hexnumber@\csname sym#2\endcsname
    {\hexnumber@{\count\z@}\hexnumber@{\count\tw@}}}
\end{Verbatim}
