%%^^J%%  um-doc-style.tex  -  part of UNICODE-MATH <github.com/wspr/unicode-math>

% /©
%
% Copyright 2006-2017   Will Robertson <will.robertson@latex-project.org>
% Copyright 2010-2013 Philipp Stephani <st_philipp@yahoo.de>
% Copyright 2012-2015     Khaled Hosny <khaledhosny@eglug.org>
%
% This package is free software and may be redistributed and/or modified under
% the conditions of the LaTeX Project Public License, version 1.3c or higher
% (your choice): <http://www.latex-project.org/lppl/>.
%
% This work is "maintained" by Will Robertson.
%
% ©/

\makeatletter
% \iffalse
% !TEX TS-program = XeLaTeX
% ^^A %%%%%%%%%%%%%%%%%%%%%%%%%%%%%%%
% ^^A   SELF-EXTRACTION BEGINS HERE
% ^^A %%%%%%%%%%%%%%%%%%%%%%%%%%%%%%%
%<*internal>
\begingroup
\input l3docstrip.tex\relax\keepsilent
\declarepreamble\defaultpreamble

Copyright 2006-2017   Will Robertson <will.robertson@latex-project.org>
Copyright 2010-2013 Philipp Stephani <st_philipp@yahoo.de>
Copyright 2012-2015     Khaled Hosny <khaledhosny@eglug.org>

This package is free software and may be redistributed and/or modified under
the conditions of the LaTeX Project Public License, version 1.3c or higher
(your choice): <http://www.latex-project.org/lppl/>.

This work is "maintained" by Will Robertson.
\endpreamble
\nopostamble
\askforoverwritefalse
\generate{\file{unicode-math.sty}{\from{unicode-math.dtx}{base}}}
\gdef\DTXFILES{%
  \DTX{unicode-math.dtx}%
  \DTX{unicode-math-preamble.dtx}%
  \DTX{unicode-math-pkgopt.dtx}%
  \DTX{unicode-math-msg.dtx}%
  \DTX{unicode-math-usv.dtx}%
  \DTX{unicode-math-setchar.dtx}%
  \DTX{unicode-math-mathtext.dtx}%
  \DTX{unicode-math-main.dtx}%
  \DTX{unicode-math-fontopt.dtx}%
  \DTX{unicode-math-fontparam.dtx}%
  \DTX{unicode-math-mathmap.dtx}%
  \DTX{unicode-math-mathtext.dtx}%
  \DTX{unicode-math-epilogue.dtx}%
  \DTX{unicode-math-primes.dtx}%
  \DTX{unicode-math-sscript.dtx}%
  \DTX{unicode-math-compat.dtx}%
  \DTX{unicode-math-alphabets.dtx}%
}
\ifx\UMDEBUG\undefined\def\UMDEBUG{}\else\def\UMDEBUG{,debug}\fi
\def\DTX#1{\from{#1}{package,XE\UMDEBUG}}
\generate{\file{unicode-math-xetex.sty}{\DTXFILES}}
\def\DTX#1{\from{#1}{package,LU\UMDEBUG}}
\generate{\file{unicode-math-luatex.sty}{\DTXFILES}}
\def\tempa{plain}\ifx\tempa\fmtname\endgroup\expandafter\bye\fi
\endgroup
\ProvidesFile{unicode-math.dtx}
%</internal>
%<base>\ProvidesPackage{unicode-math}
%<package&XE>\ProvidesPackage{unicode-math-xetex}
%<package&LU>\ProvidesPackage{unicode-math-luatex}
%<*base|package>
  [2017/10/09 v0.8h Unicode maths in XeLaTeX and LuaLaTeX]
%</base|package>
%<*internal>
\input{unicode-math-doc}
%</internal>
% \fi
%
% \clearpage
% \part{Package implementation}
% \parttoc
%
% \section{The \texttt{unicode-math.sty} loading file}
%
% The prefix for \pkg{unicode-math} is \texttt{um}:
%    \begin{macrocode}
%<@@=um>
%    \end{macrocode}
%
% The plain sty file is a stub which loads necessary packages and then bifurcates into
% a XeTeX- or LuaTeX-specific version of the package.
%
%    \begin{macrocode}
%<*base>
%    \end{macrocode}
% Bail early if necessary.
%    \begin{macrocode}
\ifdefined\XeTeXversion
  \ifdim\number\XeTeXversion\XeTeXrevision in<0.9998in%
    \PackageError{unicode-math}{%
      Cannot run with this version of XeTeX!\MessageBreak
      You need XeTeX 0.9998 or newer.%
    }\@ehd
  \fi
\else\ifdefined\luatexversion
  \ifnum\luatexversion<64%
    \PackageError{unicode-math}{%
      Cannot run with this version of LuaTeX!\MessageBreak
      You need LuaTeX 0.64 or newer.%
    }\@ehd
  \fi
\else
  \PackageError{unicode-math}{%
    Cannot be run with pdfLaTeX!\MessageBreak
    Use XeLaTeX or LuaLaTeX instead.%
  }\@ehd
\fi\fi
%    \end{macrocode}
%
% \paragraph{Packages}
% Assuming people are running up-to-date packages.
%    \begin{macrocode}
\RequirePackage{expl3,xparse,l3keys2e}
\RequirePackage{fontspec}
\RequirePackage{ucharcat}
\RequirePackage{fix-cm} % avoid some warnings (still necessary? check...)
\RequirePackage{filehook}
%    \end{macrocode}
% \paragraph{Bifurcate}
%    \begin{macrocode}
\ExplSyntaxOn
\sys_if_engine_luatex:T { \RequirePackageWithOptions{unicode-math-luatex} }
\sys_if_engine_xetex:T  { \RequirePackageWithOptions{unicode-math-xetex}  }
\ExplSyntaxOff
%    \end{macrocode}
%
%    \begin{macrocode}
%</base>
%    \end{macrocode}
%
% That's the end of the base package. The subsequent packages are derived from
% the following ordered list of \texttt{dtx} files:
% \begin{multicols}{3}
% \begin{enumerate}
% \def\DTX#1{\item \texttt{#1}}
% \DTXFILES
% \end{enumerate}
% \end{multicols}
%
\endinput



\GetFileInfo{unicode-math.dtx}
\let\umfiledate\filedate
\let\umfileversion\fileversion

\CheckSum{0}
\EnableCrossrefs
\CodelineIndex
\setcounter{IndexColumns}{2}

\usepackage[svgnames]{xcolor}
\usepackage[inline]{enumitem}
\usepackage{array,booktabs,calc,enumitem,fancyvrb,graphicx,ifthen,longtable,refstyle,subfig,topcapt,url,varioref,underscore}
\setcounter{LTchunksize}{100}
\usepackage[slash-delimiter=frac,nabla=literal]{unicode-math}
\usepackage{metalogo,hologo}

\fvset{fontsize=\small,xleftmargin=2em}
\usepackage[it]{titlesec}

\setmainfont{texgyrepagella}%
 [
  Extension = .otf ,
  UprightFont = *-regular ,
  ItalicFont = *-italic ,
  BoldFont = *-bold ,
  BoldItalicFont = *-bolditalic ,
 ]
\setsansfont{Iwona}%
 [
  Scale=MatchLowercase,
  Extension = .otf,
  UprightFont = *-Regular,
  ItalicFont  = *-Italic,
  BoldFont    = *-Bold,
  BoldItalicFont = *-BoldItalic,
 ]
\setmonofont{Inconsolatazi4-Regular.otf}%
 [
  Scale=MatchLowercase,
  BoldFont=Inconsolatazi4-Bold.otf
 ]

\setmathfont{texgyrepagella-math.otf}
\setmathfont[version=xits]{xits-math.otf}
\newfontface\umfont{xits-math.otf}

\usepackage{hypdoc}
\hypersetup{linktocpage}

% work around some issue turning | into "j" inside mathsf in the definition of \Module:
% (also prettify)
\def\Module#1{{\footnotesize\color{red}$\langle$\texttt{#1}$\rangle$}}

\linespread{1.1}
\frenchspacing

\definecolor{niceblue}{rgb}{0.2,0.4,0.8}

\def\theCodelineNo{\textcolor{niceblue}{\sffamily\tiny\arabic{CodelineNo}}}

\newcommand*\name[1]{{#1}}
\newcommand*\pkg[1]{\textsf{#1}}
\newcommand*\feat[1]{\texttt{#1}}
\newcommand*\opt[1]{\texttt{#1}}

\newcommand*\note[1]{\unskip\footnote{#1}}

\let\latin\textit
\def\eg{\latin{e.g.}}
\def\Eg{\latin{E.g.}}
\def\ie{\latin{i.e.}}
\def\etc{\@ifnextchar.{\latin{etc}}{\latin{etc.}\@}}

\def\STIX{\textsc{stix}}
\def\MacOSX{Mac~OS~X}
\def\ascii{\textsc{ascii}}
\def\OMEGA{Omega}

\newcounter{argument}

\makeatletter
\g@addto@macro\endmacro{\setcounter{argument}{0}}
\makeatother

\newcommand*\darg[1]{%
  \stepcounter{argument}%
  {\ttfamily\char`\#\theargument~:~}#1\par\noindent\ignorespaces
}
\newcommand*\doarg[1]{%
  \stepcounter{argument}%
  {\ttfamily\makebox[0pt][r]{[}\char`\#\theargument]:~}#1\par\noindent\ignorespaces
}

\newcommand\codeline[1]{\par{\centering#1\par\noindent}\ignorespaces}

\newcommand\unichar[1]{\textsc{u}+\texttt{\small#1}}

\setlength\parindent{2em}

\def \MakePrivateLetters {%
  \catcode`\@=11\relax
  \catcode`\_=11\relax
  \catcode`\:=11\relax
}

\makeatother
